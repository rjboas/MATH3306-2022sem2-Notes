% !TeX TS-program = xelatex
% !TeX options = -aux-directory=Debug -shell-escape -file-line-error -interaction=nonstopmode -halt-on-error -synctex=1 "%DOC%"
\documentclass{article}
\input{LaTeX-Submodule/template.tex}

% Additional packages & macros
\geometry{a4paper}

% Header and footer
% ! Unit name
\newcommand{\unitName}{Set Theory \& Mathematical Logic}
% ! Unit code
\newcommand{\unitCode}{MATH3306}
% ! Unit semester 
\newcommand{\unitTime}{Semester 2, 2022}
% ! Unit coordinator name
\newcommand{\unitCoordinator}{Professor Benjamin Burton}
% ! Document authors
\newcommand{\documentAuthors}{Rohan Boas}

\fancyhead[L]{\unitName}
\fancyhead[R]{\leftmark}
\fancyfoot[C]{\thepage}

% Copyright
\usepackage[
    type={CC},
    modifier={by-nc-sa},
    version={4.0},
    imagewidth={5em},
    hyphenation={raggedright}
]{doclicense}

\date{}

\begin{document}

\begin{titlepage}
    \vspace*{\fill}
    \begin{center}
        \Large{\unitCode} \\[0.05in]
        \LARGE{\textbf{\unitName}} \\[0.1in]
        \normalsize{\unitTime} \\[0.2in]
        \normalsize\textit{\unitCoordinator} \\[0.2in]
        \documentAuthors
    \end{center}
    \vspace*{\fill}
    \doclicenseThis
    \thispagestyle{empty}
\end{titlepage}
\newpage

\tableofcontents
\newpage

\section{Introductory notes}
\subsection{Gödel's incompleteness theorem}
\begin{theorem}[Gödel's incompleteness theorem, baby version]
    There are true mathematical statements that cannot be proven.
\end{theorem}
\begin{proof}[``Proof'']
    Take the statement ``This statement has no proof.''

    \emph{Assume it is false.}
    This implies that the statement has a proof.
    If the statement has a proof, it must be true, contradiction!
    
    \emph{Assume it is true.}
    This implies that the statement has no proof.
    Therefore, the statement cannot be proven.
\end{proof}
\subsection{The halting problem}
The halting problem is undecidable.

TODO:\@ Add context and proof.
\subsection{Defining algorithms}
In defining algorithms, Turing machines
and recursive functions will be the primary focus.
Grammars and code are also alternatives. 
\begin{definition}[Church-Turing thesis, informal]
    Any reasonable definitions of ``algorithm'' are equivalent.
\end{definition}

\end{document}